%================================================================
\chapter{PSO}
%================================================================

Um grande estímulo no desenvolvimento de algoritmos é o projeto de algoritmos para resolver problemas cada vez mais complexos. Grande sucesso tem sido alcançado na modelagem de inteligência biológica e natural, resultando nos chamados sistemas inteligentes. Esses algoritmos inteligentes incluem:

\begin{itemize}
    \item redes neurais artificiais (sistemas neurais biológicos)
    \item computação evolucionária (evolução natural)
    \item inteligência coletiva (comportamento social de organismos)
    \item sistemas imunológicos artificiais (o sistema imunológico humano)
    \item sistemas nebulosos (interação de organismos com o meio ambiente)
\end{itemize}

Inteligência pode ser definida como a habilidade de compreender a partir da experiência, interpretar inteligência, tendo a capacidade para pensamento e razão, especialmente em alto nível, incluindo criatividade, consciência, emoção e intuição. Enquanto tem-se obtido sucesso na modelagem de pequenas partes de sistemas neurais biológicos, ainda não há solução para o problema complexo de modelar intuição, consciência e emoção, que são partes integrantes da inteligência humana.

Inteligência computacional \cite{bib:engelbrecht2007} pode ser vista como um ramo da Inteligência Artificial que estuda os mecanismos adaptativos para permitir ou facilitar um comportamento inteligente em ambientes complexos ou mutáveis. Inteligência Coletiva (\textit{Swarm Intelligence} - SW) tem sua origem no estudo de colônias ou enxames de organismos sociais. Otimização por exame de partículas (\textit{Particle Swarm Optimization} – PSO) \cite{bib:kennedyeberhart1995} é um dos exemplos de estratégias de inteligência coletiva, inspirada em bando de aves em busca por alimento.

PSO é um procedimento de busca baseado em população, onde os indivíduos, referidos como partículas, são agrupados em um enxame e cada partícula representa uma solução candidata ao problema de otimização. Em um PSO, cada partícula é lançada pelo espaço de busca multidimensional, ajustando sua posição de acordo com sua experiência e a das partículas vizinhas, fazendo uso da melhor posição por ela encontrada e da melhor posição encontrada pelas vizinhas para se movimentar na direção da solução ótima. O desempenho de cada partícula é medido de acordo com uma função de aptidão prédefinida, que está relacionada ao problema a ser resolvido. Cada partícula atualiza sua posição adicionando frações de três deslocamento:

\begin{itemize}
    \item uma fração do deslocamento na mesma direção que estava seguindo no passo anterior;
    \item uma fração do deslocamento na direção da posição onde registrou o melhor desempenho até o momento;
    \item uma fração do deslocamento na direção da posição da vizinha com o melhor desempenho no momento.
\end{itemize}

No algoritmo denominado Global Best (Gbest), a vizinhança de cada partícula é formada por todas as demais partículas do enxame.

\begin{verbatim}
Algoritmo Global Best PSO
    Crie e inicialize um exame com n partículas;
    Repita
        para i := 1 a n faça
            Calcule a aptidão da partícula;
            se aptidãoi <= pbesti então
                Atualize pbest com a nova posição da partícula;
            se pbesti <= gbest então
                Atualize gbest com a nova posição;
                Atualize a velocidade da partícula;
                Atualize a posição da partícula;
    até Condição de parada;
    Retorne Melhor resultado.
\end{verbatim}

Gráficos para ilustrar